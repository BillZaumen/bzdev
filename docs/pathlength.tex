\documentclass[12pt]{article}
\title{Computing the length of a B\'{e}zier path}
\author{Bill Zaumen}
\date{October 17, 2023}
\addtolength{\textheight}{5.0ex}

\begin{document}
\maketitle

\section{Introduction}

B\'{e}zier paths, as used in practice, consist of multiple
segments, each of which is either a straight-line segment, a
quadratic B\'{e}zier curve segment, or a cubic B\'{e}zier
curve segment. These curves can be represented by the following
expressions:
\begin{itemize}
  \item $\mathbf{p}(t) = \mathbf{p}_0B_{0,1}(t) + \mathbf{p}_1B_{1,1}(t)$
        for linear segments,
  \item $\mathbf{p}(t) = \mathbf{p}_0B_{0,2}(t) + \mathbf{p}_1B_{1,2}(t)
	+ \mathbf{p}_2B_{2,2}(t)$
	for quadratic segments,
  \item $\mathbf{p}(t) = \mathbf{p}_0B_{0,3}(t) + \mathbf{p}_1B_{1,3}(t)
	 + \mathbf{p}_2B_{2,3}(t) + \mathbf{p}_3B_{3,3}(t)$
	 for cubic segments,
\end{itemize}
where $\mathbf{p}_0$, $\mathbf{p}_1$, $\mathbf{p}_2$, and $\mathbf{p}_0$
are control points,  $B_{i,n}(t)$ is a Bernstein polynomial defined
by the equation
\[ B_{i,n}(t) = \left(\begin{array}{c}n \\ i\end{array}\right)
   t^i(1-t)^{n-i},\]
with $\mathbf{p}(t)$ and the control points represented by vectors.

If the X and Y components of $\mathbf{p}(t)$ are $x(t)$ and $y(t)$ (with
an obvious generalization to more dimensions), then the distance between
two points on a segment is
\[ d(t_1,t_2) = \int_{t=t1}^{t2}\sqrt{A(t)^2 + B(t)^2}\,dt \]
where
\begin{eqnarray*}
  A(t) & = & \frac{dx}{dt} \\
  B(t) & = & \frac{dy}{dt}
  \end{eqnarray*}
These derivatives can be computed by using the relation,
\[ \frac{d}{dt}B_{i,n}(t) = n(B_{i-1,n-1}(t) - B_{i,n-1}(t)) \]
with $B_{-1,n-1}(t) = B_{n,n-1}(t) = 0$.  This relation can be
used to differentiate $x$ and $y$, representing the derivatives
as polynomials using a B\'{e}zier basis and the derivatives
can then be easily transformed to a monomial basis.

Because $A(t)$ and $B(t)$ are squared, the square root will
always be positive unless $A(t)$ and $B(t)$ share a common
root.  In addition, $A(t)$ and $B(t)$ can be at most quadratic
polynomials. When $A(t)$ and $B(t)$ share at least one root,
we can write the integral as
\[ d(t_1,t_2) = \int_{t=t1}^{t2}|P(t)|\sqrt{Q(t)}\,dt\]
where P(t) and Q(t) are polynomials and where the degree of
$Q$ is either 2 or 0.
      
\section{Straight line segments} \label{linear}

For linear line segments, $A(t)$ and $B(t)$ are constants and the
integral is trivial: the length of a line segment is a constant
multiplied by $t_2 - t_1$, with positive values indicating that the
path parameter is increasing and negative values indicating that the
path parameter is decreasing.

\section{Quadratic line segments} \label{quadratic}

For quadratic line segments, $A(t)$ and $B(t)$ are constants or
first degree polynomials.  If $A(t)$ and $B(t)$ have a common
root, the integrand is the absolute value of a
polynomial. Otherwise, the integral can be written as
\[ d(t_1, t_2) = \int_{t=t_1}^{t_2} \sqrt{a + bt + ct^2}\,dt\]
That integral can be found in the CRC Standard Mathematical
Tables, which states that it is
\[ d(t_1,t_2) = \left. \frac{(2ct+b)\sqrt{X}}{4c}
\right|_{t=t_1}^{t_2} + \frac{1}{2k}\int_{t=t_1}^{t_2}\frac{1}{
\sqrt{X}}\,dt\]
where $q = 4ac-b^2$, $k = \frac{4c}{q}$ and $X = a + bt + ct^2$
In addition,
$\int_{t=t_1}^{t_2}\frac{1}{\sqrt{X}}\,dt$ has different values for
\begin{itemize}
  \item $c > 0$, in which case
	\[ \int \frac{dt}{\sqrt{X}} = \frac1{\sqrt{c}}
	\log (\sqrt{X} + t\sqrt{c} + \frac{b}{2/sqrt{c}}) \]
  \item $c > 0$ and $q < 0$, in which case
	\[ \int \frac{dt}{\sqrt{X}} = \frac1{\sqrt{c}}
	\sinh^{-1} \left(\frac{2ct+b}{\sqrt{4ac-b^2}}\right) \]
  \item $c < 0$, in which case
	\[\int \frac{dt}{\sqrt{X}} = \frac1{\sqrt{-c}}
	\sin^{-1} \left(\frac{-2ct-b}{\sqrt{b^2 - 4ac}}\right) \]
\end{itemize}
For $d(t_1,t_2)$, there are a few special cases.  If $q = 0$, the
integral reduces to the integral of the absolute value of a
first degree polynomial and that is a good approximation when q
is very close to zero as well.  Also, when $|c|$ is sufficiently
small, one can change the integral to
\begin{eqnarray*}d(t_1,t_2) & = & \int_{t=t_1}^{t_2} \sqrt{a+bt}
\sqrt{1 +\frac{ct^2}{a+bt}}\,dt \\
& \approx & \int_{t=t_1}^{t_2}\left[
 \sqrt{a+bt} + \frac{ct^2}{2\sqrt{a+bt}}
\right]\,dt
\end{eqnarray*}
and both terms in the approximate value have integrals in the CRC
Standard Mathematical Tables:
\begin{eqnarray}
  \int \sqrt{a + bt}\,dt & = & \frac{2}{3b}\sqrt{(a+bt)^3} \\
  \int \frac{t^2}{\sqrt{a+bt}}\,dt & = &
       \frac{3(8a^2 - 4abt + 3b^2t^2)}{15b^3}\sqrt{a+bt}
\end{eqnarray}

For integrals containing absolute values of polynomials, it is
necessary to break the integral up into segments whose boundaries
are the limits of integration and the roots, with no segment including
a root internally. Finally, when the integral uses logarithms, it is
possible for $a+bt+ct^2$ to be positive for $t\in[t_1,t_2]$ and the
argument for the logarithm to be negative.  For this case
use $\log x = \log (-x) + i\pi$ for $x < 0$. While the integral is
a complex number, the imaginary part is a constant and thus drops out
of definite integrals.

\section{Cubic line segments}\label{cubic}

For cubic line segments, $A(t)$ and $B(t)$ could be constants or
first degree polynomials, but are typically second degree
polynomials If $A(t)$ and $B(t)$ have a common root, the
integrand is the square root of a constant or a second degree
polynomial $Q(t)$, multiplied by the absolute value of a polynomial
$P(t)$. There are three cases:
\begin{enumerate}
  \item if $A(t)$ and $B(t)$ share one common root, $P(t)$ is a first
	degree polynomial and $Q(t)$ is a second degree polynomial
  \item if $A(t)$ and $B(t)$ share two roots or both have the same
	double root, then $P(t)$ is a second-degree polynomial and
	$Q(t)$ is a constant.
  \item If there are no roots in common, $Q(t) = A(t)^2 + B(t)^2$ is a
	fourth degree polynomial with no real roots, and can be
	factored into two polynomials $Q_1(t)$ and $(Q_2(t)$ that
	have no real roots and that have positive values for all values
	of $t$.
\end{enumerate}

Cases 1 and 2 involve integrals that are listed in the CRC Standard
Mathematical Tables, and use elementary functions. While each has
several cases, these are similar to the ones outlines in
Section~\ref{quadratic}.  One additional integral is useful:
      \[ \int x\sqrt{X}\,dt = \frac{X\sqrt{X}}{3c}
        - \frac{b}{2c}\int\sqrt{X}\,dt \]
where $X = a + bt + ct^2$. This integral can be used for the case
where $A(t)$ and $B(t)$ have on common root, in which case the
integral is that for the absolute value of a first degree polynomial
multiplied by the square root of a quadratic polynomial.

The third case, which requires integrating the square root of
a quartic polynomial, uses elliptic integrals. If $Q_1(t)$ and
$Q_2(t)$ are written as
\begin{eqnarray*}
Q_1(t) & = & f_1 + g_1t+h_1t^2 \\
Q_2(t) & = & f_2 + g_2t + h_2t^2 ,
\end{eqnarray*}
the integral can be found in paper by B. Carlson,
\emph{A Table of Elliptic Integrals: Two Quadratic Factors}.\footnote{
  B.C. Carlson,
  "A Table of Elliptic Integrals: Two Quadratic Factors,"
  Mathematics of Computation, Volume 59, Number 199, July 1992,
  Pages 165--180.
  $<$https://www.ams.org/journals/mcom/1992-59-199/S0025-5718-1992-1134720-4/S0025-\\
5718-1992-1134720-4.pdf$>$
}
Equation 2.45 on Page 169 of Carlson's paper, labeled as
[1, 1, 1, 1], contains this integral, and makes use of a number of
variables Carlson defined before Equation 2.45. One obtains
\begin{eqnarray*}
\int_{t=y}^{x}\sqrt{Q_1(t)Q_2(t)}\,dt & = &
     (\delta^2_{22}/h_2^2 - \delta_{11}^2/h_1^2)[\psi_0H_0
     + (\Delta_0 - \delta_{12}^2)R_f]/8 \\
& - & (3\psi_0^2-4h_1h_2\delta_{12}^2)(\Sigma+\delta_{12}^2R_f)
  / (24h_1^2h_2^2) \\
& + &  [\Delta^2R_f-\psi_0A(1,1,1,1)]/(12h_1h_2) + E/(3h_1)
\end{eqnarray*}
where the quantities in this integral are defined below. In the following
list of definitions, $i$ has each of two values (1 and 2) and the
functions $R_F$, $R_J$, $R_C$, and $R_D$ are the Carlson symmetric
forms of elliptic integrals:
\begin{eqnarray*}
  H_0 & = &  \delta_{11}^2\psi_0 [
			R_J(M^2,L_{-}^2,L_{+}^2,\Omega_0^2)/3
			+R_C(a_0^2,b_0^2)/2]/h_1^2 \\
	&  &		-X_0R_C(T_0^2,V_0^2) \\
  a_0^2 & = & b_0^2 + \Lambda_0(\Lambda_{+} - \Lambda_0)
			(\Lambda_0 - \Lambda_{-}) \\
  b_0^2 & = &  (S^2/U^2 + \Lambda_0)\Omega_0^4 \\
  V_0^2 & = & \mu_0^2(S^2 + \Lambda_0U^2)  \\
  T_0 & = & \mu_0S + 2h_1h_2  \\
  \mu_0 & = & h_1/(\xi_1\eta_1)  \\
  X_0 & = & -(\xi_1^\prime\xi_2 + \eta_1^\prime\eta_2)/(x-y) \\
  \Omega_0^2 & = & M^2 + \Lambda_0  \\
  \Lambda_0 & = &  \delta_{11}^2h_2/h_1 \\
  \psi_0 & = & g_1h_2 - g_2h_1 \\
  A(1,1,1,1) & = & \xi_1\xi_2 - \eta_1\eta_2  \\
  S & = & (\xi_1\eta_1\theta_2 + \xi_2\eta_2\theta_1)/(x-y)^2 \\
  \Sigma & = & G-\Delta_{+}R_f + B  \\
  R_f & = & R_F(M^2,L_{-}^2, L_{+}^2)  \\
  G & = & 2\Delta\Delta_{+}R_D(M^2,L_{-}^2,L_{+}^2)/3 + \Delta/(2U) \\
    & &	 (\delta_{12}^2\theta_1 - \delta_{11}^2\theta_2)/(4\xi_1\eta_1U) \\
  L_{\pm}^2 & = & M^2 + \Delta_{\pm}  \\
  \Delta_{\pm} & = & \delta_{12}^2 \pm \Delta \\
  \Delta &=& (\delta_{12}^4 -\delta_{11}^2\delta_{22}^2)^{1/2} \\
  \delta_{ij} & = & (2f_ih_j + 2f_jh_i - g_ig_j)^{(1/2)}  \\
  M & = & \zeta_1\zeta_2/(x - y) \\
  U & = & (\xi_1\eta_2+\eta_1\xi_2)/(x - y)  \\
  \zeta_i & = & [(\xi_i+\eta_i)^2 - h_i(x-y)^2]^{(1/2)} \\
  \theta_i & = & \xi_i^2 + \eta_i^2 - h_i(x-y)^2  \\
  E & = & \xi_1^\prime\xi_1^2\xi_2-\eta_1^\prime\eta_1^2\eta_2 \\
  B & = & \xi_1^\prime\xi_2 - \eta_1^\prime\eta_2 \\
  \eta_1^\prime & = & (g_1 + 2h_1y)/(2\eta_1)  \\
  \xi_1^\prime & = & (g_1 + 2h_1x)/(2\xi_1)  \\
  \eta_i& = & (f_i + g_iy + h_iy^2)^{1/2}  \\
  \xi_i & = & (f_i + g_ix + h_ix^2)^{1/2}
\end{eqnarray*}
Programs that define these variables should place them in the
reverse order: the list uses the convention in which quantities
are used before they are defined, the opposite of that used by Carlson.

An algorithm for factoring $Q(t)$ is described
in the article \emph{Factoring Quartic Polynomials: A Lost Art}.\footnote {
    Gary Brookfield, "Factoring Quartic Polynomials: A Lost Art",
    Factoring Quartic Polynomials: A Lost Art</A>",
    Mathematics Magazine, Vol. 80, No. 1, February 2007, Pages 67--70,
    https://www.maa.org/sites/default/files/Brookfield2007-103574.pdf
}
The easiest way to proceed is to factor $Q(t)$ into three terms:
a constant, and two quadratic polynomials whose $t^2$ coefficients are
1.0. There are two special cases.
\begin{enumerate}
  \item the minimum value of at least one of the two factors is
	small (e.g., less than $1 \times 10^{-4}$). In this case,
	tests indicate that numerical accuracy is poor, and one should
	use some other method such as numeric integration.
  \item the two factors are identical.  In this case, the integrand
	is the absolute value of a quadratic polynomial. For this case,
	the integral given in Carlson's paper can fail: a Java implementation
	returned Double.NaN, most likely because of division by zero,
	which suggests that Carlson's integral may not be numerically
	accurate when the two factors are nearly, but not exactly
	identical.
\end{enumerate}
When the two factors are nearly identical (for example, when the
coefficients differ by less than $10^{-7}$), one can write
\begin{eqnarray*}
\sqrt{Q_1(t)Q_2(t)} & = & Q_1(t)\sqrt{\frac{Q_2(t)}{Q_1(t)}}
 =  Q_1(t)\sqrt{1 + \frac{Q_2(t)}{Q_1(t)} - 1} \\
& \approx & Q_1(t)\left[1 + \frac12 (\frac{Q_2(t)}{Q_1(t)} - 1)
\right]
= \frac{Q_1(t) + Q_2(t)}{2}
\end{eqnarray*}
and the integral can be approximated by the integral of a polynomial.
\end{document}

% LocalWords:  zier dt dx dy monomial integrand bt CRC sqrt abt pdf
% LocalWords:  quartic ij ih jh ig iy Brookfield NaN
